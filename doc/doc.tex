\documentclass{article}
\usepackage{amsmath}
\usepackage{amssymb}

\begin{document}

\section{Purpose}
Many cartoons of semiconductor device cross-sections are sufficiently simple that routines can be made for their construction. This package provides utilities for scripting the creation of semiconductor device cross-sections. 

\section{Coordinate System}
The global Cartesian coordinate system has origin at the upper left hand corner of the schematic. Cartesian subdomains of the graphics canvas, are rectangular in shape with origin in their lower left hand corner.

\section{Description of Objects}\label{sec:oop}

The objects are, in ascending order of class hierarchy (in the sense of which objects contain lower objects, not which objects are subclasses of which objects), features, layers, and devices. The objects are immutable and contain both mutable and immutable attributes. The layer object creates many features provided one or no feature, and the device object modifies layers and the features they contain. Both devices and layers are effectively lists with construction methods and attributes. One can access the desired layers through device items or features through layer items as well as the references (named variables) assigned to them.

Objects each have a place method, which calls the place methods of all objects that are contained. This is done so that any object can have its drawing instructions made and added to the canvas, and so that the cumulative translation from placement of each container is easily calculated.

\subsection{Schematic}
The schematic is a collection of devices and only is used to layout the partially created devices in sequence for diagrams illustrating semiconductor fabrication processes.

\subsection{Device}
The device is a collection of layers and has no parameters which directly affect the drawing of features. Rather it maintains a stack height on which to deposit another layer. Layers may be stacked in a list or successively. Those layers which are stacked in a list will be stacked at the same height (overlapping). This means the stack requires a memory since the unstack operation won't know which layers were stacked as a list and which were singly stacked. This memory of stack heights is kept in the same collection object as the layers. 

In addition to a stack which defines the height at which each layer is drawn, there may generally be a different stack desired which defines the drawing precedence. This is in cases where, instead of clipping features, it may be desired to draw over their parts. This is not implemented, but if the entries in the stack are permuted (which is easily achieved by list indexing, though the stack data structure is bad for permuting), it may be achieved.

\subsection{Layer}

As defined, a layer has the attributes of phase, period, domain, and feature. The period is here synonymous with pitch. The phase is best reported as a phase fraction varying from 0 to 1. The start of the first feature $c$, the phase fraction $\phi$, and the period $p$ are related by

$$ \phi = \bmod(c, p) \,.$$

The device width restricts features on some $0 < x < w_d$. For features partially inside the device but whose widths extend out of it clipping is used. One may prefer to specify starting location of a feature, which is related to a phase shift by $x_0 = p \phi$. One may specify either $x_0$ or $\phi$ for the layer. If both $x_0$ and $\phi$ are specified, the program is permissive and assumes they add in contribution.

Note the layer is not a collection of features, unlike both the schematic and devices, since it is defined as infinite in the $x$-direction. The place method takes the device width and returns the features within a bounding box.

\subsection{Feature}

A feature has a shape and size. Subclasses of the PolygonFeature construct the coordinates provided shapes and sizes for common shapes. The ConvexPolygon feature allows arbitrary coordinates to be specified provided they form a convex polygon. 

\subsubsection{Polygon Features}
Polygons under the constraint that no two points share the same phase angle and the shape is closed around the origin, are easily generally accounted for by ordering the coordinates by polar angle $\theta = \arctan2(x,y)$ and then drawing lines between these ordered coordinates. Some routines such as the algorithm for conformal layer may depend on polygons being convex.

\subsubsection{Non-linear features}
Non-linear features require vector graphics programs to be efficiently rendered. Standard non-linear shapes such as circles and ellipses are supported by the vector graphics programs. In general, numerical algorithms approximating these as composed of polygons are required. 

\section{Clipping}
A layer has a defining box (=subdomain). The coordinates of a layer are defined relative to a lower left hand corner of a layer box for which features outside are clipped. The $x$-limits are a device attribute (width) and the $y$-limits are a layer attribute (height), since the layer is defined as infinite in the $x$ direction. The layer height is taken as the feature height unless explicitly given.

\section{Copying} 
%the copy methods as written are deep copy methods because they copy all the references recursively, e.g., copying a device copies also the layers and also the features
Each object has a copying method defined for it. Because features are repeated their copying is used in layer creation. For creating conformal layers it is easiest to copy a given layer and apply a transform to all its features (which are also copied in a deepcopy). Devices are copied for making schematics which show the creation of a device by a sequence of steps.

\section{Convenience Functions for Symmetric Placement of Features in Layers}

If the device width $w_d$ is not equal to $np + w_f$ where $n$ is an integer and $w_f$ is the feature width, it will not be symmetric. The offset $x_0$ required is calculated from the balance equation

$$x_0 + \frac{w_f}{2} + np + \frac{w_f}{2} + x_0 = w_d\,, $$

Here $n$ is also an unknown number of periods in the device width while able to keep a positive $x_0$ (within the device width). That is, the solution is for $\max n$ subject to $x_0 > 0$. One may simplify the equation to 

$$\frac{2x_0}{p} = \frac{w_d - w_f}{p} - n \,. $$

From which evidently $n = \lfloor (w_d - w_f)/p \rfloor$.

A features spacing $s_f$ between neighboring features, which is equal to the feature padding, is related to the period of the layer it is in $p$ and the feature width $w_f$ by

$$p = w_f + s_f \,.$$

Hence provided a desired feature width and spacing, the correct period and offset for symmetry can be calculated.

One may also want to put in a number of features and a feature width and calculate the required period to fit those features bordering the edge of the device on both sides. Then the period is related to the layer width, feature width, and number of features by

$$ w_l = (n-1)p + w_f\,. $$

One may also want to have a given number of features, but the features on either end are split symmetrically. This is simple to evaluate the period for: where $n$ is the total number of features, regardless whether they are clipped or not, $p = w_d/(n-1)$ and $x_0 = -w_f/2$. Note the period in this case is independent of feature width. The total number of features observed is actually $n-1$ since the features at either end are cut in half.

\section{Conformal Layers}

Given $\mathbf{p}_i$, $\mathbf{p}_{i+1}$, \ldots where $\mathbf{p}_i$ is the vertex which forms an edge with $\mathbf{p}_{i-1}$ and $\mathbf{p}_{i+1}$, there are the following definitions:

\begin{enumerate}
    \item The perpendicular bisector of the edge $e_{i+1}$ defined between $\mathbf{p}_i$ and $\mathbf{p}_{i+1}$ is a line which intercepts the average of the two vertices, $\mathbf{b}_i = (\mathbf{p}_i + \mathbf{p}_{i+1})/2$, and has as slope $$\mathbf{m}_i = \mathbf{R}\mathbf{d}_{i+1}$$ where $\mathbf{d}_{i+1} = \mathbf{p}_{i+1} - \mathbf{p}$ is the vector with length and slope of the edge and $\mathbf{R}$ is the rotation matrix 2-D, that is, $\mathbf{R} \equiv ((0,1),(-1,0))$
    \item The angular bisector of the vertex $\mathbf{p}_i$, denoted $a_i$, is a line which intercepts $\mathbf{b}_i' = \mathbf{p}_i$ and has as slope 
        $$ \mathbf{m}_i' = (\mathbf{p}_i - \mathbf{p}_{i+1}) + (\mathbf{p}_i - \mathbf{p}_{i-1})  = 2\mathbf{p}_i - \mathbf{p}_{i+1} - \mathbf{p}_{i-1} \,. $$
\end{enumerate}

The perpendicular bisector $e_{i+1}$ intercepts the angular bisector $a_i$ at one point and forms a triangle with the other two intercepts already known for the perpendicular bisector and the angular bisector on the polygon perimeter. The equation for the intercept is

$$ x_i \mathbf{m}_i + \mathbf{b}_{i} = x_i' \mathbf{m}_{i+1}' + \mathbf{b}_i'\,.$$

Then there is a linear equation to be solved for the extent along each line, $\mathbf{x}_i$

$$ \begin{pmatrix} \mathbf{m}_i & -\mathbf{m}_{i+1}' \end{pmatrix} \mathbf{x}_i = \mathbf{b}_i' - \mathbf{b}_i \,.$$

    The intercept position $\mathbf{y}_i$ is easily calculated from this by substituting into either of the starting linear equations the relevant extent along that slope. In order to create a conformal geometric layer, that triangle should be magnified to a similar triangle. The angle may be found as 

$$\cos\theta = \frac{ (\mathbf{b}_i - \mathbf{y}_i) \cdot (\mathbf{b}_i' - \mathbf{y}) }{ \Vert \mathbf{b}_i - \mathbf{y}_i \Vert \Vert \mathbf{b}_i' - \mathbf{y}_i \Vert } \,.  $$

The coordinate of the conformal layer is given by some constant increment from the perpendicular bisector intercept outward along the perpendicular bisector. Its other coordinate is then given by that increment scaled by the trigonometric factor from the angle bisector intercept outward along the angle bisector. This assumes that the intercept is always interior to the polygon, and that the distance to the vertex is greater than the distance to the perpendicular bisector. The edge should always form a leg of the triangle, the line segment from $\mathbf{y}_i$ to $\mathbf{b}_i$ another leg, and the line segment from $\mathbf{y}_i$ to $\mathbf{b}_i'$ the hypotenuse.

For irregular shapes the distance along the vertex has to be different for the two neighboring perpendicular bisectors to create a conformal layer of a fixed thickness. One may desire that the conformal layer appears like a continuous film. Consider $\theta_i$ as the one formed between a perpendicular bisector $e_i$ and an angular bisector $a_i$, and $\theta_{i+1}$ as the one formed between a perpendicular bisector $e_{i+1}$ and an angular bisector $a_i$. The increments from the vertex at $a_i$ are weighted as $\sec\theta$, so that if without loss of generality $\theta_i < \theta_{i+1}$, then $\sec(\theta_i) < \sec(\theta_{i+1}$ and it is the conformal coordinate associated with the perpendicular bisector $e_{i+1}$ that is farther out. Then it is possible to, along the line defined between $\tilde{\mathbf{b}}_i$ and $\tilde{\mathbf{b}}_i'$, where a tilde indicates the conformal coordinate analog, move the coordinate so that it is collinear with the line defined between $\tilde{\mathbf{b}}_{i-1}$ and $\tilde{\mathbf{b}}_i''$ (double prime indicating it is generally different, since the factors will differ). This is equivalent to finding the intercept of the two lines and using that as the common conformal coordinate $\tilde{\mathbf{b}}_i'$ for the angular bisector $a_i$. 

\end{document}
